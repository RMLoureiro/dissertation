%!TEX root = ../template.tex
%%%%%%%%%%%%%%%%%%%%%%%%%%%%%%%%%%%%%%%%%%%%%%%%%%%%%%%%%%%%%%%%%%%%
%% chapter6.tex
%% NOVA thesis document file
%%
%% Chapter with a short latex tutorial and examples
%%%%%%%%%%%%%%%%%%%%%%%%%%%%%%%%%%%%%%%%%%%%%%%%%%%%%%%%%%%%%%%%%%%%

\typeout{NT FILE chapter6.tex}%

\chapter{Conclusions}\label{cha:conclusions}


\section{Summary}\label{sub:summary}

Blockchain technologies have seen a rise in popularity in recent years, giving rise To
numerous blockchain protocols and consensus algorithms. The development of
such protocols involves a lot of decisions regarding their behavior and parameterizations,
and reasoning about these issues is often not trivial, especially without an environment to
test and validate their execution.

Having a simulator that can provide information about the protocol's execution and a wide
range os scenarios to test, provides a great tool for prototiping and validating
blockchain protocols with more confidence in their implementation. MOBS can also
be a powerful debugging tool since we can use it to reproduce specific scenarios in a
repeatable and controlled environment, which can be very difficult in real world executions.

We validated MOBS capabilities by implementing and testing several well known
and documented consensus algorithms and confirmed that we can extract enough
qualitative data to validate the protocols' correctness. We can also see that is possible
to extract quantitative data that can be used to compare different protocols and
parameterizations, and even if the execution times are not representative of a real
world scenario, we can still use them as a comparison for different protocols in
the sae scenario.

The possibility of creating a more parameterizable network layer that can
simulate different network and membership protocols to provide scenarios more in line
with real world scenarios. Even if our attempt was not successful, we believe that
given a different approach, it is possible to achieve this goal.


\section{Future Work}\label{sub:future_work}

Our work can be further expanded in several ways:

\begin{itemize}
    \item \textbf{Network Layer Improvements:} The network layer can be improved to
    provide more realistic network conditions. This could provide the user a selection
    of different network protocols to better simulate different real world scenarios.

    \item \textbf{Protocol validation layer:} The simulator can also be expanded with a network
    module that can validate the protocol's correctness, allowing the user when parameterizing
    the execution to select which properties to validate during the execution, or what validation
    script should be executed at the end of the simulation. This would allow a more streamlined
    execution instead of having to run external scripts after the simulator has finished.

    \item \textbf{Seperate implementation for Byzantine nodes:} The current implementation
    of MOBS does not provide a way to implement Byzantine nodes that run concurrently with
    regular nodes. The current way to achieve Byzantine behavior is to implement the Byzantine
    code concurrently with the regular code and defining a priori which nodes will be Byzantine.
    This leads to a more complex implementataton and a greater risk of deviating from the
    original protocol.

    \item \textbf{Ability to stop and resume simulations:} The current simulator only allows
    for the parameterization of the simulation at the start of the execution, then the user needs
    to wait for the execution to finish in order to validate the protocol's execution. Adding a
    debugger like process to stop the execution of the simulation under certain conditions and allow
    the user to inspect the state of the nodes would be a powerful tool to analyze more complex scenarios
    where the messages might not provide enough information to understand the root of the problem.
\end{itemize}

