%!TEX root = ../template.tex
%%%%%%%%%%%%%%%%%%%%%%%%%%%%%%%%%%%%%%%%%%%%%%%%%%%%%%%%%%%%%%%%%%%
%% chapter1.tex
%% NOVA thesis document file
%%
%% Chapter with introduction
%%%%%%%%%%%%%%%%%%%%%%%%%%%%%%%%%%%%%%%%%%%%%%%%%%%%%%%%%%%%%%%%%%%

\typeout{NT FILE chapter1.tex}%

\chapter{Introduction}\label{cha:introduction}

This thesis aims to address practical challenges in designing, implementing and maintaining blockchain consensus
protocols by providing a simulation environment to analyze their behavior and offering empirical metrics of their runtime in
different environments and conditions. To achieve this we take the Modular Blockchain Simulator (\href{https://github.com/mce-alves/MOBS
}{MOBS}) and expand the previously done work to help better test and validate these protocols with tools to extract statistics and 
qualitative data about their runtime. We also tested the viability to provide a more dynamic and independent network layer
to better simulate real-world conditions, this provides programmers a tool to more quickly prototype protocols or solutions
in a simulated, modular and parametrizable environment where executions can be repeated and a wide range of scenarios
can be used for testing.

\prependtographicspath{{Chapters/Figures/Covers/}}

\section{Context}\label{sub:context}
With each passing day the amount of distributed applications like E-banking and social networks increases, these applications
leverage state machine replication to offer distributed and reliable services. A subset of these are applications
that have strict requirements about the integrity of its data, transactional operations by authenticated users, 
and simultaneous access for updating and
consulting the records. For this specific subset there are a group of protocols that have been created to meet and ensure
these requirements. Blockchain protocols allow for simultaneous access, integrity check and update of records across
a distributed database, each node has its own copy of the ledger that it uses to keep data integrity and reach a consensus
about its accuracy.

Around 2008, blockchain protocols appeared with the motivation of 
a distributed ledger for cryptocurrency transactions, they offer decentralization enhanced security and transparency given that
the history of transactions usually being public. These protocols are not static, being because vulnerabilities,
flaws that need to be corrected or changes proposed by the stake-holders.
One of these dynamic protocols is Tezos~\cite{tezos}, which relies on the stake-holders that participate in the system to propose and
agree on changes and upgrades to the protocol. These changes are done with Tezos' self-amendment, which enables the network to undergo
changes without the need for a network fork, which in most blockchains is the common practice.
Another example is Ethereum~\cite{ethereum} that started by using a blockchain protocol based on proof
of work and in 2022 migrated towards an implementation based solely on proof of stake for Ethereum2. This change opened
Ethereum to bouncing attacks that hindered the finalization of blocks\cite{ethereum_analysis}.
Due to the nature of data that these handle, errors and vulnerabilities like these can be costly.
This opens a necessity for tools that aid in support these evolutions in a faster, more seamless and secure fashion.

Blockchain protocols operate on top of membership layers that dictate how the topology of the network is configured.
Different membership environments come with different properties and trade-offs. Structured membership overlays allow for faster lookups for
specific nodes and a pre-defined and predictable structure to the network. Non-structured membership offer a more resilient overlay when
new nodes are introduced or existing ones leave, albeit by choice or failure. And overlays that operate by building a partially structured
overlays allow for the benefits of a non-structured overlay at the cost of slower re-structuration since optimization
procedures are regularly executed to improve routing and lookup operations. The trade-offs and some of these protocols are further
explained in Chapter~\ref{cha:Background}.

The challenges in developing blockchain protocols motivated the development of MOBS, a modular and extensible simulator that provides
the ability to simulate different families of blockchain and consensus protocols.
MOBS provides a parametrizable execution of the selected protocols, exhibiting a modular and extensible structure and offers detailed logs for the
qualitative evaluation for the study of implemented protocols. However, this simulator has limitations such as a network layer
that only allows for static parameterizations, network layout and node behaviors are defined before runtime and there is no mechanism to
re-structure the network after a node fails or leaves the system, there is also a lack of qualitative data making it difficult to
evaluate the protocols' execution.

\section{Problem}\label{sub:problem}
Consensus protocols are inherently complex to design and implement correctly~\cite{paxos, have_we_reached_consensus}.
In blockchain systems, where financial transactions and dynamic behavior are the primary features, protocol correctness is essential,
as even minor errors can lead to substantial financial losses and system malfunctions. One illustration of this is the transition of
Ethereum from a Proof of Work to a Proof of Stake consensus. This transition exposed the protocol to vulnerabilities that could be
exploited by bouncing attacks on liveness~\cite{ethereum_analysis}.
The chain is unable to be finalized as a result of this type of attack, as the primary selected chain in the fork
choice rule is perpetually hopping between two alternative branches.

There is also Solana~\cite{solana}, a new blockchain protocol that relies on Proof-of-History to build its chain,
where repeated testing results showed that the protocol does not fully achieve consensus and
a single malicious validator can halt the Solana blockchain~\cite{solana_halting_problem}. These tests also showed that there are
inconsistencies in the behavior between what is described in the documentation and what the protocol showed since Solana's implementation
has deviated in undocumented ways from the available protocol design descriptions.

Consensus and blockchain protocols are usually described with pseudocode and model checked with idealized languages that 
not reflect the implementations. Since validating correctness is very costly~\cite{desidn_and_validation}, before planning an
implementation, developers should test their ideas and prototypes. There is a lack of methodologies to develop,
evaluate, tune and replicate these protocols, opening a need for extensible and modular tools for consensus analysis and
experimental labs for rapid prototyping.
With this in mind MOBS was developed with the aim to give the ability to test and validate these protocols under different conditions and settings by 
changing the execution parameters, aiming to catch vulnerabilities or even logic errors before deploying changes to these protocols.
Since verification of protocols is very costly, developers can use MOBS to get confidence in their implementations
from experimentations first.

Right now the parameters to the network are set before the execution of the simulation, and regarding the network behavior, we can
set the network topology and how many nodes will fail and when. In the real world a network's topology is dynamic, new nodes can enter
as new participants, existing ones can leave, either by choice or by failure, and even when the participants are static the network
can suffer changes to its topology as a result of optimizations performed by this layer.

Currently, in MOBS there is no dynamic parameterization of the network layer, making it hard to simulate real world conditions,
together with the lack of qualitative data provided by the simulator, validating a protocol's executions becomes hard.


\section{Goal}\label{sub:goal}

With the different properties that different network overlays provide besides the parameterizations that are already provided by MOBS,
we can offer a more complete simulator that allows for the study of the behavior in different membership protocols.
This will allow us to leverage the modularity of this layer to better simulate the behavior of new participants coming into the system, existing
ones leaving and how changes to the network topology of the network affects their execution.

Another aspect we improve is the logging module of MOBS, by providing a template for logging of consensus protocols and
an after execution analysis tool to validate their execution and extract metrics for
comparison between executions in different environments, like average time to reach consensus, consensus agreement percentage or
number of agreements messages received after consensus.


The goal of this thesis is twofold: first to improve the networking layer of MOBS by making it more modular and therefore giving the following advantages:
\begin{itemize}
  \item Different environments for more diverse testing scenarios,
  \item Stronger parameterization regarding the behavior of the network,
  \item Better qualitative data by providing scenarios that better mimic real world execution;
\end{itemize}

\noindent~and secondly improve the logs to provide better qualitative data so that we can quickly analyse the
protocols properties and check their execution, allowing the programmer to spot early flaws or even vulnerabilities
by extensive simulation, gaining confidence in the execution of the protocol before moving to formally prove their correctness. 
Thus, we provide concise, meaningful and transversal data like:
\begin{itemize}
  \item Average time to consensus,
  \item Quorum agreement percentage for each consensus,
  \item Number of late agreement messages;
\end{itemize}

This allows users to better evaluate the protocols' execution and combined with the more modular and more
parameterizable simulator provide a better environment to quickly prototype protocols and solutions while testing in
a wide range of parameterizable scenarios.

To achieve this we extended MOBS (\href{https://github.com/RMLoureiro/MOBS}{MOBS-Fork}) in the following ways:
\begin{itemize}
  \item Implement and study what properties are needed to evaluate the execution of consensus protocols, these include Paxos, Chandra-Toueg, PBFT and Ethereum.
  \item Evaluate the execution of blockchain protocols and replicate known
    vulnerability scenarios, namely Ethereum Probabilistic Bouncing attack~\cite{ethereum_analysis}.
\end{itemize}



\section{Contributions}\label{sub:contributions}
The contributions to this thesis focuses on the implementation and evaluation of consensus protocols in MOBS, to verify that MOBS can be used
has a tool for prototyping and validation. The protocols implemented and evaluated were:
\begin{itemize}
  \item Chandra-Toueg~\cite{chandra}, a simple consensus protocol to ensure the logs from MOBS can be used to evaluate consensus properties,
  \item Paxos~\cite{paxos}, a more complex consensus protocol that allows for greater flexibility in the face of network partitions,
  \item PBFT~\cite{pbft}, a practical Byzantine fault-tolerant protocol that is widely used in permissioned blockchain systems,
  \item Ethereum~\cite{ethereum}, a decentralized platform that enables the creation of smart contracts and decentralized applications and specifically
replicated the Probabilistic Bouncing attack and the patch that was implemented validating its executions.
\end{itemize}